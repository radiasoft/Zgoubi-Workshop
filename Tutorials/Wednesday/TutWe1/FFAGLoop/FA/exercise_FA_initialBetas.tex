\documentclass[10pt]{article}

\usepackage{multirow}
\usepackage{rotating,graphicx}
%\usepackage{wrapfig}
\usepackage{amssymb}
\usepackage{amsmath}
\usepackage{lscape}
\usepackage{times}
\usepackage{color}  % For \textcolor and \color % Ex. : \textcolor{red}{Text colored with} ; {\color{red}Text colored with}
\usepackage{soul}   % For \hl{ highlighted text} ; \sethlcolor{colorname}
%\usepackage[table]{xcolor}
\usepackage{xcolor,colortbl}
\usepackage{capt-of}
\usepackage{textcomp}  % allows \textonehalf,  \textonequarter, or else
%\usepackage{mathcomp}  % make it math compatible

\oddsidemargin =-0.6in
\evensidemargin=-0.6in
%\textwidth=7.6in   % owl
 \textwidth=7.8in % laptop
\textheight=10.2in
% \topmargin=-.25in   % owl
 \topmargin=-1.in    % laptop
\footskip=-.0in

\newcommand{\Bl}{\ensuremath{B_{{low}}}}
\newcommand{\Bh}{\ensuremath{B_{{high}}}}
\newcommand{\Il}{\ensuremath{I_{{low}}}}
\newcommand{\Ih}{\ensuremath{I_{{high}}}}
\newcommand{\Br}{\ensuremath{B\! \rho}}
\newcommand{\CE}{concentration ellipse}
\newcommand{\CES}{CE-$\rm \mathcal{S}$}
\newcommand{\dEE}{\small \ensuremath{\frac{dE}{E}}}
\newcommand{\eg}{\ensuremath{\it e.g.}}
\newcommand{\Ha}{\ensuremath{\mathcal{H}}}
\newcommand{\Sa}{\ensuremath{\mathcal{S}}}
\newcommand{\vs}{\ensuremath{\it vs.}}
\newcommand{\ie}{\ensuremath{\it i.e.}}
\newcommand{\hbrk}{\hfill \break}
\newcommand{\sr}{synchrotron radiation}
\newcommand{\SRl}{SR loss}
\newcommand{\x}{\ensuremath{x}} 
\newcommand{\xp}{\ensuremath{{x'}}}
\newcommand{\z}{\ensuremath{z} }
\newcommand{\zp}{\ensuremath{{z'}}}
\newcommand{\y}{\ensuremath{y} }
\newcommand{\yp}{\ensuremath{{y'}}}
\newcommand{\dl}{\ensuremath{{\delta l}}} 
\newcommand{\dE}{\ensuremath{{\delta E}}}
\newcommand{\X}{\ensuremath{X} }
\newcommand{\Xp}{\ensuremath{{X'}}}
\newcommand{\Z}{\ensuremath{Z} }
\newcommand{\zg}{Zgoubi}
\newcommand{\Zp}{\ensuremath{{Z'}}}
\newcommand{\Y}{\ensuremath{Y} }
\newcommand{\Yp}{\ensuremath{{Y'}}}
\newcommand{\Len}{\ensuremath{l} }
\newcommand{\Mom}{\ensuremath{E}}
\newcommand{\Sx}{\ensuremath{\mathcal{S}_x}}
\newcommand{\Sy}{\ensuremath{\mathcal{S}_y}}
\newcommand{\Sz}{\ensuremath{\mathcal{S}_z}}

\newcommand{\C}{\ensuremath{\mathcal{C}}}
\newcommand{\D}{\ensuremath{\mathcal{D}}}
\newcommand{\HH}{\ensuremath{\mathcal{H}}}
\newcommand{\bHH}{\bar \HH}
\newcommand{\com}{{center of mass}}
\newcommand{\lab}{{laboratory frame}}
\newcommand{\LL}{\ensuremath{\mathcal{L}}}
\newcommand{\rms}{\ensuremath{rms}}
\newcommand{\wrt}{{with respect to}}

\newcommand{\bull}{\ensuremath{\bullet~}}
\newcommand{\cf}{\ensuremath{\textsl{cf.}}}
\newcommand{\bhel}{\ensuremath{\mathbf{^3He^{2+}}}}
\newcommand{\hel}{\ensuremath{\mathrm{^3He^{2+}}}}
\newcommand{\nib}{\noindent \ensuremath{\bullet~}}
\newcommand{\snib}{\noindent {\small \ensuremath{\bullet~}}}
\newcommand{\nid}{\noindent \ensuremath{\diamond~}}
\newcommand{\snid}{\noindent {\small \ensuremath{\diamond~}}}
\newcommand{\nin}{\noindent~}
\newcommand{\no}{\ensuremath{\mathbf{\vec n_0}}}
\newcommand{\MC}{Monte~Carlo}
\newcommand{\p}{\ensuremath{\mathbf{p}}}
\newcommand{\pp}{$\rm p\! \! \uparrow$}

\definecolor{orange}{rgb}{1,0.5,0}
\definecolor{yelloworange}{rgb}{1,.647,0}
\newcommand{\black}{\color{black}}
\newcommand{\red}{\color{red}}
\newcommand{\green}{\color{green}}
\newcommand{\blue}{\color{blue}}
\newcommand{\yo}{\color{yelloworange}}

\newcommand{\referenceA}{\rm  }
\newcommand{\referenceB}{\rm }
\newcommand{\referenceC}{\rm }

\pagestyle{headings}
\markboth{\small  \referenceA ~ ~   \referenceB ~ ~  \referenceC \hfill }
         {\small  \referenceA ~ ~   \referenceB ~ ~  \referenceC \hfill }


\begin{document}

\thispagestyle{empty}

\begin{minipage}{1.\linewidth}
\bf
  \flushright{F. M\'eot}
\vspace{-2ex}
  
  \flushright{BNL C-AD}
\vspace{-2ex}
  
  \flushright{Zgoubi 2019 Workshop, Boulder, CO}
\vspace{-2ex}
  
\flushright{24-29 Aug. 2019} 
\end{minipage}


\vspace{5ex}

\centerline{\LARGE \bf
  Simulation of CBETA BNL-CORNELL ERL Using Field Maps
}

~

\centerline{\LARGE \bf
1. Periodic Orbits and Optical Functions 
}

\centerline{\LARGE \bf
  Along the Permanent Magnet Return Loop
}

\vspace{5ex}
\author{
F.~M\'eot
\\
Collider-Accelerator Department, BNL, Upton, NY 11973 \\
}


The inetrest of using field maps is that it brings greatest accuracy in the modeling of the optical elements (strictly speaking,
of their magnetic field). Allied with stepwise ray-tracing techniques, it also brings greatest accuracy on
the resolution of the equation of particle motion through these fields. 

We will lean on Ref.~\footnote{CBETA Design Report, J.~Barley et als., Jan~27, 2017.} 
for the details concerning CBETA design and its permanent magnet return loop, 
and on Refs.~\footnote{A Full Field-Map Modeling of Cornell-BNL CBETA 4-Pass Energy Recovery Linac, F.~M\'eot et als., ICAP18, Key West.}~\footnote{Beam dynamics validation of the Halbach Technology FFAG Cell for Cornell-BNL Energy Recovery Linac, F. Méot et als., NIM A, vol. 896, pp. 60-67, 2018.} concerning the present field map based optical methods. 
In particular, the theoretical arc cell is displayed in [1]~Fig.~2.2.3, p.~24 or in [2]~Fig.~3; the geometry of the ERL and of its return loop are shown in [1]~Figs.~2.1.1, p.~18 and 2.1.2, p.~19 respectively. The OPERA model of the permanent magnet cell is shown in [2]~Figs.~6 and~11.

There is no quadrupole knob over the FA-TA-ZA-ZB-TB-FB return loop, on the other hand there is corrector dipole windings
(Fig.~11 in [2]) but we will ignore them here. 
In these conditions the reference orbits and the optical functions (betatron functions and dispersion)
along the loop are fully determined by their values at the start of the FA arc.
This exercise deals with the determination of the latter.
Tests will be performed by propagating the initial orbit and optical function values so determined, 
all the way along the $\rm \approx 50$\,m loop.




\section*{Working hypotheses:}

There are two subtelties in the handling of the periodic optical functions of the permanent magnet return loop.
>>>>>>> .r1247

<<<<<<< .mine
\smallskip

\nib There are two subtelties in the handling of the periodic optical functions of the permanent magnet return loop.

\nin - First, while the QF-BD doublet FA cell repeats itself 16 times over the upstream FA arc of the return loop (Fig.~2.1.2, p.~19, in [1]), the arc actually starts with a half BD magnet.
This is for the purpose of optical matching at the connection between S1 and the FA arc. 
||||||| .r1248
\nib First, while the QF-BD doublet FA cell repeats itself 16 times over the upstream FA arc of the return loop (Fig.~2.1.2, p.~19, in [1]), the arc actually starts with a half BD magnet.
This is for the purpose of optical matching at the connection between S1 and the FA arc. 
=======
Thus, the optical functions upstream of that half-cell, which is equivalently the downstream end of S1, are determkined by the periodic functions of
the FA cell.
>>>>>>> .r1247

Thus, the optical functions upstream of that half-BD, at the location of the downstream end of S1, are determkined by the periodic functions of the FA cell.

\nib Next, an aspect proper to the use of field maps:  the S1-FA arc connexion happens to be located within the extent of
the half-BD field map, thus getting them from there requires a numerical method.

We choose here to use a FIT procedure as a convenient and simple way to achieve that.



\section*{ Numerical experiments: }


\nin 1/ Find the 4 design energy orbits in the FA cell, namely: reproduce the plot of Fig.~5-left in [2]. 

Plot the magnetic field along the 4 orbits.

~

\nin 2/ Find the periodic functions at the end of the FA cell,
for the 4 design energies, using \texttt{TWISS}.

~

\nin 3/ Add the half-BD section that ensures the connection to S1,
upstream of the FA cell. Using the optical functions at the downstream end of that sequence as
the constraints, find the optical functions at the S1-FA connection.  \texttt{FIT[2]} can be used for that. 

~

\nin 4/ Check your results:

- inject the initial orbit coordinates at the start of the FFAG loop (file FFAGLoop.dat),
plot them, all 4 energies (in a similar manner to the 42\,MeV orbit in [2], Fig.~13-left)

- inject the initial optical functions at the start of the FFAG loop, 
for instance for the 42\,MeV case, and verify that they propagate correctly all the way. 

 

\end{document}
